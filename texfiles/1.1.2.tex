\documentclass[11pt]{article}
\usepackage{amsmath, amssymb, enumerate, mathtools, dsfont, MnSymbol}

\begin{document}


{\bfseries Section 1.1.2}

\begin{enumerate}[1]
    \item %1 
        graph $g$ of order $n$ with maximal number of edges if the complete graph
        $k_n$, which has $\binom{n}{2} = \frac{n(n-1)}{2}$ edges.
    \item %2
        To show a contradiction, suppose otherwise. We know there are an even 
        number of odd degree vertices, implying that there can be no odd 
        degree vertices. But the max degree of a graph is n-1 (connected to 
        every other vertex), so the degree is between 0 and n-1. Excluding 
        the odd degrees, there are not enough unique numbers to cover all the
        vertices.
    \item %3 
        For later.
    \item %4  
        If no such path exists, the two odd vertics are on separate connected 
        components A and B. Consider A by itself, it is a connected graph, 
        but it has an odd number of vertices with an odd degree, a contradiction
        . 
    \item %5
        \begin{enumerate}[a)]
            \item 
                
            \item 
        \end{enumerate}
    \item %6
        We prove this by induction on the length of the odd closed walk. 

        Base Case: The length 3 odd closed walk is just a length 3 cycle.

        Suppose odd closed walks with lengths up to 2n - 1 contain odd cycles.
        Let $W$ = $v_1$, $v_2$, ..., $v_{2n+1} = v_1$ be a length $2n+1$ closed walk. 
        If no vertices in the walk repeat, then we are done, the odd walk is an
        odd cycle. Otherwise, let $l$ be the smallest number not $1$ such that $v_l$ 
        repeats, and let $v_l = v_k$ where $l < k$. Then we have two closed walks 
        in $W$, $v_1$, ..., $v_l = v_k$, ..., $v_{2n+1} = v_1$ and $v_l$, ..., 
        $v_k = v_l$. The lengths of these two walks must add up to $2n+1$, thus one
        of them must be an odd length closed walk, which by the inductive hypothesis
        must contain an odd length cycle. 


    \item %7

    \item %8
        

    \item %9
        My guess is that it is a complete graph $K_n$ with every edge to a single 
        vertex missing, which has $\binom{n-1}{2} = \frac{(n-2)(n-1)}{2}$

    \item %10
        (Not sure whether this proof is legitimate/rigorous).

        Using induction we prove the the minimum number of edges needed to have 
        a connected graph is $n - 1$ (a tree). 

        Base Case: $K_1$ has $0$ edges (certainly the minimum number), and is connected.

        Suppose $G$ of order $n$ requires minimum ${n - 1}$ edges to be connected.
        Consider such a connected graph with $n - 1$ edges, and consider $G + {v}$,
        $G$ with an additional vertex. If we add no edges, 

    \item %11
        $\Rightarrow$ Suppose $e = uv$ is a bridge of $G$ and consider $G-e$. Since
        $e$ is a bridge, $G-e$ is disconnected, i.e. no $uv$ path exists in $G-e$,
        implying that $e$ is not part of any cycle in $G$. 


        $\Leftarrow$ Suppose $e = uv$ is not a bridge. Then $G - e$ is still connected, 
        i.e. there exists a $uv$ path. This path $+ e$ is a cycle.

    \item %12
        \begin{enumerate}[a)]
            \item 
            \item 
            \item
        \end{enumerate}

    \item %13
        

    \item %14
        Suppose $G$ has no cycles and is connected. Consider $v \in V$ such that 
        $deg(v) > 1$ and $x,y \in N(v)$. Graph $G - \{v\}$ is disconnected since 
        no $xy$ path exists since if there was, this path in addition to 
        path $x$, $v$, $y$ would form a cycle in $G$. If $order = 2$, the graph does 
        not have a cycle. 
    \item %16


\end{enumerate}
\end{document}
