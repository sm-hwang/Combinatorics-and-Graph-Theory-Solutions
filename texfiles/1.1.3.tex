\documentclass[11pt]{article}
\usepackage{amsmath, amssymb, enumerate, mathtools, dsfont, MnSymbol}

\begin{document}


{\bfseries Section 1.1.3}

\begin{enumerate}[1]
    \item %1 
        We have an edge for every distinct pairs of vertices, so
        $\binom{n}{2} = \frac{n(n-1)}{2}$.
    \item %2
        We prove the contrapositive. Suppose $r_1 \neq r_2$, $|X| = r_1$ and 
        $|Y| = r_2$. Then for $x \in X$, $deg(x) = r2$ and for $y \in Y$ 
        $deg(y) = r1$, but $r2 \neq r1$, so $K_{r_1, r_2}$ is not regular.
    \item %3 
        No, no matter which $4$ vertices you choose, $2$ of them will be in the 
        same subset, and would have no edges between them. 
    \item %4  
        No. Having all $4$ vertices in the same set will induce a disconnected
        graph, $3$ in $X$ and $1$ in $Y$ would result in a edges between the 
        single vertex in a partition set to the other three, and $2$ in $X$ 
        and $2$ in $Y$ induces a graph with a cycle.
    \item %5
        
    \item %6
        
    \item %7
        For part c), we have $order = m$, $size = \sum^{n}_{i=1} 
        \frac{(r_i - 1)r_i }{2}$.

    \item %8
        Suppose $G$ and $H$ are isomorphic. There exists a bijection $f$ from 
        $V(G)$ to $V(H)$ such that $xy \in E(G) \Leftrightarrow f(x)f(y) \in E(H)$. 
        $$xy \in E(\overline{G}) \Leftrightarrow xy \notin E(G) \Leftrightarrow f(x)f(y) \notin E(H) \Leftrightarrow
        f(x)f(y) \in E(\overline{H})$$. Thus the complements are isomorphic.

        
\end{enumerate}
\end{document}
