\documentclass[11pt]{article}
\usepackage{amsmath, amssymb, enumerate, mathtools, dsfont, MnSymbol}

\begin{document}


{\bfseries Section 1.3.3}

\begin{enumerate}[1]
    \item %1
        Let $G$ be a connected graph. Assign weight 1 to all edges of $G$, and 
        run Kruskal's algorithm. The algorithm produces a minimal spanning tree,
        thus $G$ contains at least one minimum spanning tree.
    \item %2
        $\Rightarrow$ Suppose $G$ is a tree. Remember that $|E(G)| = n - 1$ 
        where $n = |V(G)|$. $G$ is connected by definition,
        and contains all its vertices so $G$ is a spanning tree of $G$ which 
        contains all its edges. If $G$ contained more than one spanning tree, 
        (other than $G$ itself), than these two spanning trees must differ 
        on at least one edge, meaning $|E(G)| > n - 1$, a contradiction. 
        
        $\Leftarrow$ Suppose a graph $G$ is connected and contains exactly one 
        spanning tree $S$. We show that $G = S$. $S$ is a subgraph of $G$, thus
        $V(S) \subseteq V(G)$ and $E(S) \subseteq E(G)$. If there exists 
        $v \in V(G), v \notin V(S)$, then $S$ is not a spanning tree of 
        $G$, thus $V(G) \subseteq V(S)$. Suppose $\exists e \in E(G), e \notin 
        E(S)$. We give weight of $0$ to $e$, and weight $1$ to all other $e' \in
        E(G)$, and run Kruskal's algorithm, producing a minimum spanning tree
        $T$. $T \neq S$, since $e \in T$ since the algorithm will always choose 
        the lowest weight edge, but $e \notin S$ by assumption, leading to 
        a contradiction that $G$ has only one minimum spanning tree. Thus
        $E(G) \subseteq E(S)$, and $G = S$. 

    \item %3
        Let $T$ be a spanning tree of $G$. Suppose $\overline{T}$ does not contain
        any edges in $C$. Then $T$ contains $C$, thus $T$ is not a tree. 

    \item %4
        We prove the contrapositive of the two statements.

        $\Rightarrow$ Suppose $e \in E(G)$ and 
        there exists a spanning tree $T$ such that $e \notin E(T)$. Then 
        $T$ spans $G - e$, meaning $\forall u,v \in V(G), \exists uv$ path in 
        $T$, so $G - e$ is connected.

        $\Leftarrow$ Suppose $e$ is not a bridge. Then $G - e$ is connected,
        and thus has a spanning tree $T$. This is a spanning tree of $G$ that 
        doesn't contain $e$. 
\end{enumerate}
\end{document}
