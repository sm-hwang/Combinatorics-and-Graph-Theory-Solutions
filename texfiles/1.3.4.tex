\documentclass[11pt]{article}
\usepackage{amsmath, amssymb, enumerate, mathtools, dsfont, MnSymbol}

\begin{document}


{\bfseries Section 1.3.4}

\begin{enumerate}[1]
    \item %1
        Prufer's sequence records the neighbour of the smallest leaf and deletes
        the leaf, thus the only way for a leaf to be recorded is if it is the 
        neighbour of the smallest leaf left, but then the graph at this point 
        is $K_2$ and the algorithm should stop. 

        Vertex $v$ is recorded if they are the neighbour of a leaf with the
        minimum label, and the leaf is deleted afterwards. Thus the number
        of times $v$ can be in the seqeunce is $ < deg(v)$. We want to show 
        that this number is $ > deg(v) - 2$. Suppose it is $ \leq deg(v) - 1$. 
        The same number of neighbours of $v$ is removed as the number of its
        appearance in the sequence, thus the resulting tree must have $x, y$ 
        such that $x \neq y$ and $xv, yv \in E(T)$. But this resulting tree
        is not $K_2$, thus the algorithm should not have stopped. 

    \addtocounter{enumi}{2}
    \item %4
        Whenever a leaf is chosen, it must all have the same neighbour, implying
        that a star is such a tree. From 1, and knowing the Prufer sequences
        are $n - 2$ long, we must have that a single vertex $v$ in such a graph 
        must have $deg(v) = n - 2 + 1 = n - 1$. Trees have $n - 1$ edges, so
        every edge is incident with this $v$, thus a star is the only graph with 
        this property.
    \item %5
        Whenever a leaf is chosen, it must have a distinct neighbour to the ones
        already seen in the sequence. Paths are such trees with the property. 
        Exercise 1 implies that the maximum degree of such a tree is $2$. We 
        know every tree must have at least $2$ leaves. To show that paths $(P_n)$
        are the only type of trees with this property, we show that having more 
        than $2$ leaves results in $\Delta(G) > 2$. Suppose such a tree had more 
        than $2$ leaves, say $x, y$ and $z$. Since trees are connected, we know
        there is a $x-y$ path $(P_v = v_1, ..., v_m)$ and $x-z$ path 
        $(P_u = u_1, ..., u_k)$. Let $i, j$ be minimum indices such that 
        $v_i = u_j$. Then $v_i$ has degree at least $3$, since $y \notin P_u$ 
        and $z ]notin P_v$ because they are leaves, $v_i \neq x$ since $x$ is 
        a leaf, and $v_iv{i-1}, v_iv_{i+1}, v_iu_{j+1}\in E(G)$, where
        $v_{i+1} \neq u_{i+1}$. 

    \item %6
        From: \textit{https://math.stackexchange.com/questions/666997/how-many-
        \newline different-spanning-trees-of-k-n-setminus-e-are-there}

        $K_n$ contains $\binom{n}{2}$ edges, and each tree of order $n$ contains
        $n - 1$ edges, meaning each spanning tree has $\frac{n - 1}{
        \binom{n}{2}} = \frac{2}{n}$ of all edges. Equivalently, any edge belongs
        to $\frac{2}{n}$ of all spanning trees. $1 - \frac{2}{n}$ trees don't 
        contain any edge $e$ and by Cayley's theorem, $\frac{n-2}{n}n^{n-2} 
        = (n-2)n^{n-3}$ don't contain $e$. 

    \item %7
        We apply the Matrix Tree Theorem to $K_n$.

        For $K_n$, we have 
        \[
            D - A = \begin{bmatrix}
                    n-1 & - 1 & \dots & -1 \\\
                    -1 & n-1 & & \vdots\\
                    \vdots & & \ddots & \\
                    -1 & \dots & -1 & n-1 \\
                    \end{bmatrix}
        \]
        Thus,
        \[
            cof_{1,1}(D - A) = det(\begin{bmatrix}
                    n-1 & - 1 & \dots & -1 \\\
                    -1 & n-1 & & \vdots\\
                    \vdots & & \ddots & \\
                    -1 & \dots & -1 & n-1 \\
                \end{bmatrix})
        \]
        where the above matrix is a $(n-1) x (n-1)$ matrix. Doing appropriate
        row and column operations, we have 
        \[
            cof_{1,1}(D - A) = det(\begin{bmatrix}
                    1 & - 1 & \dots & -1 \\\
                    0 & n &  & \vdots\\
                    \vdots &  & \ddots & \\
                    0 & \dots & 0 & n \\
                \end{bmatrix})
        \]
        and the determinant of an upper triangular matrix is the product of 
        the diagonal entries, thus $cof_{1,1}(D - A) = n^{n-2}$. 
        

\end{enumerate}
\end{document}
