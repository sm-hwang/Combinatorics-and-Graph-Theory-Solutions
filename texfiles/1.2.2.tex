\documentclass[11pt]{article}
\usepackage{amsmath, amssymb, enumerate, mathtools, dsfont, MnSymbol, tikz}
\usepackage{algorithm}
\usepackage[noend]{algpseudocode}
\begin{document}


{\bfseries Section 1.2.2}

\begin{enumerate}[1]
    \item %1 
        We have an edge for every distinct pairs of vertices, so
        $\binom{n}{2} = \frac{n(n-1)}{2}$.
    \item %2
        We prove the contrapositive. Suppose $r_1 \neq r_2$, $|X| = r_1$ and 
        $|Y| = r_2$. Then for $x \in X$, $deg(x) = r2$ and for $y \in Y$ 
        $deg(y) = r1$, but $r2 \neq r1$, so $K_{r_1, r_2}$ is not regular.
    \item %3 
        No, no matter which $4$ vertices you choose, $2$ of them will be in the 
        same subset, and would have no edges between them. 
    \item %4  
        \begin{enumerate}[a)]
            \item
                $[A^3]_{j,j}$ equals the number of length $3$ walks from $v_j$ to 
                itself. Length $3$ closed walks form a triangle that contains 
                $v_j$, but the walk $v_j$, $v_x$, $v_y$, $v_j$ and 
                $v_j$, $v_y$, $v_x$, $v_j$ are both counted. These two walks form the 
                same triangle, so we must divide the entry by two. 
            \item
                a) implies that $\frac{1}{2}\text{Tr}(A^3)$ equals the number of 
                triangles that contains $v_1$ or $v_2$ or ... or $v_n$. However 
                every triangle consists of three vertices, so we are counting 
                every triangle 3 times. Thus the number of unique triangles is 
                $\frac{1}{6}\text{Tr}(A^3)$.
        \end{enumerate}
    \item %5
        This entry is $0$. To show this, we prove that no odd length walk exists from 
        $1$ to any vertex labeled by an odd number. This is certainly true for walk of 
        length $1$. Suppose it is true for $2n - 1$, and to show a contradiction 
        suppose it is not true for $2n + 1$. Without loss of generality, suppose that 
        $5$ is the vertex that is reached. The only length $2$ walks from $5$ are 
        from $5$ to $3$, $5$ to $7$ or $5$ to $5$, thus the walk must consist of one 
        of these, and a length $2n - 1$ walk. However no $2n + 1 - 2 = 2n - 1$ length
        walk exists from $1$ to $3$, $5$ or $7$ by hypothesis. 

    \item %6
        \begin{enumerate}[a)]
            \item
                Let $i$ be the row with all positive entries in $S_r$. 
                From 1. of Theorem 1.9, we know $ecc(v_i) = r$. Since for $a < r$,
                no row had all positive entries, none of the eccentricity for other
                vertices in $G$ are less than $r$, so $r$ is the smallest 
                eccentricity, the radius of $G$. 
            \item
                Let $i$ be the row with all positive entries in $S_m$, but not in
                $S_{m - 1}$. From 1. of Theorem 1.9, $ecc(v_i) = m$, and since 
                $S_{m-1}$ contained zeros, all other eccentricities are $ \leq m$,
                thus $diam(G) = m$. 
        \end{enumerate}

    \item %7
        \begin{algorithm}
            \caption{Finding center of graph}
        \begin{algorithmic}[1]
            \State S = $[0]$, n by n zero matrix
            \State k = 0
            \While{everyrow of $S$ has a $0$}
            \State S += $A^k$
            \State k += $1$
            \EndWhile
            \State return k 
        \end{algorithmic}
        \end{algorithm}

    \item %8
        We prove the contrapositive. Suppose $G$ was not complete. $\exists v_i,v_j 
        \in V(G)$ such that $uv \notin E(G)$, meaning $A_{i,j} = 0$. If there exists
        a $v_iv_j$ path, then $D_{i, j} > 0$. Otherwise, $D_{i, j} = \inf$. In either
        case, $D_{i, j} \neq 0$, thus $A \neq D$. 

        
\end{enumerate}
\end{document}
